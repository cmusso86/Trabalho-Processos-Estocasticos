
%%%
% TIPO DE DOCUMENTO E PACOTES ----
%%%%
\documentclass[12pt, a4paper, twoside]{article}
\usepackage[left = 3cm, top = 3cm, right = 2cm, bottom = 2cm]{geometry}

$if(highlighting-macros)$
$highlighting-macros$
$endif$

\usepackage[brazilian]{babel}
\usepackage[utf8]{inputenc}
\usepackage{booktabs}
\usepackage{amsmath, amsfonts, amssymb}
\numberwithin{equation}{subsection} %subsection
\usepackage{fancyhdr}
\usepackage{graphicx}
\usepackage{colortbl}
\usepackage{titletoc,titlesec}
\usepackage{setspace}
\usepackage{indentfirst}
%\usepackage{natbib}
\usepackage[colorlinks=true, allcolors=black]{hyperref}
%\usepackage[brazilian,hyperpageref]{backref}
\usepackage[alf]{abntex2cite}
\usepackage{multirow} % https://www.ctan.org/pkg/multirow
%\\usepackage{float} % https://www.ctan.org/pkg/float
\usepackage{booktabs} % https://www.ctan.org/pkg/booktabs
\usepackage{enumitem} % https://www.ctan.org/pkg/enumitem
\usepackage{quoting} % https://www.ctan.org/pkg/quoting
\usepackage{epigraph}
\usepackage{subfigure}
\usepackage{anyfontsize}
\usepackage{caption}
\usepackage{adjustbox}
\usepackage{bm}
\usepackage{floatrow}
\floatsetup[figure]{capposition=bottom}
\floatsetup[table]{capposition=top}




\raggedbottom % https://latexref.xyz/_005craggedbottom.html


% COMANDOS -----
%%%%

\newtheorem{teo}{Teorema}[section]
\newtheorem{lema}[teo]{Lema}
\newtheorem{cor}[teo]{Corolário}
\newtheorem{prop}[teo]{Proposição}
\newtheorem{defi}{Definição}
\newtheorem{exem}{Exemplo}

\newcommand{\titulo}{$titulo$}
\newcommand{\subtitulo}{$subtitulo$}
\newcommand{\autora}{$aluno1$}
\newcommand{\autorb}{$aluno2$}
\newcommand{\autorc}{$aluno3$}
\newcommand{\autord}{$aluno4$}
\newcommand{\orientador}{ $orientador$ }
\usepackage[labelsep=period,labelfont=bf
]{caption}




\pagestyle{fancy}
\fancyhf{}
%\renewcommand{\headrulewidth}{0pt}
\setlength{\headheight}{16pt}
%C - Centro, L - Esquerda, R - Direita, O - impar, E - par
\fancyhead[RO, LE]{\thepage}
\renewcommand{\sectionmark}[1]{\markboth{#1}{}}

\titlecontents{section}[0cm]{}{\bf\thecontentslabel\ }{}{\titlerule*[.75pc]{.}\contentspage}
\titlecontents{subsection}[0.75cm]{}{\thecontentslabel\ }{}{\titlerule*[.75pc]{.}\contentspage}

\setcounter{secnumdepth}{3}
%\setcounter{tocdepth}{3}

\DeclareCaptionFormat{myformat}{ \centering \fontsize{10}{12}\selectfont#1#2#3}
\captionsetup{format=myformat}

%%%
%% INÍCIO DO DOCUMENTO 
%%%%%%

%% CAPA ----
\begin{document}
\begin{titlepage}
\begin{center}
\begin{figure}[h!]
	\centering
		\includegraphics[scale = 0.8]{img/unb.png}
	\label{fig:unb}
\end{figure}
{\bf Universidade de Brasília \\
\bf Instituto de Exatas \\
\bf Departamento de Estatística}
\vspace{5cm}

\setcounter{page}{0}
\null
\Large
\textbf{\titulo} \\
\normalsize{\emph{\subtitulo}}
\vspace{2.5cm}

\small
\vspace{0.2cm}
\textbf{\autora}\\
\textbf{\autorb}\\
\textbf{\autorc}\\
\textbf{\autord}\\

\vspace{1.5cm}
\small
Professor(a): \orientador \\
\end{center}



\vspace{5cm}

\begin{center}
{\bf{Brasília} \\ }
\bf{$ano$}
\end{center}
\end{titlepage}



\newpage

\pagenumbering{arabic}
\setcounter{page}{2}
\onehalfspacing




\setlength{\parindent}{1.5cm}
\setlength{\parskip}{0.2cm}
\setlength{\intextsep}{0.5cm}

\titlespacing*{\section}{0cm}{0cm}{0.5cm}
\titlespacing*{\subsection}{0cm}{0.5cm}{0.5cm}
\titlespacing*{\subsubsection}{0cm}{0.5cm}{0.5cm}
\titlespacing*{\paragraph}{0cm}{0.5cm}{0.5cm}

\titleformat{\paragraph}
{\normalfont\normalsize\bfseries}{\theparagraph}{1em}{}

\pagenumbering{arabic}
\setcounter{page}{3}

\fancyhead[RE, LO]{\nouppercase{\emph\leftmark}}
%\fancyfoot[C]{Departamento de Estatística}

% SUMÁRIO
%%%

\tableofcontents

\newpage


% CONTEÚDO (AS SEÇÕES SAO SEPARADAS NO RMARKDOWN) ---
%%%


$body$


%% CRONOGRAMA ----
%%%



%% REFERÊNCIAS ----
\newpage
\bibliography{$referencias$}


\end{document}

